%%% File encoding is utf8
%%% You can use special characters just like ä,ü and ñ
\mainmatter
\chapter{Einleitung}
In der Einleitung einer Abschlussarbeit sollte die Thematik der Arbeit und die Motivation, wieso dieses Thema bearbeitet wird, beschrieben und begründet werden. Der Leser soll eine Vorstellung von dem Inhalt der Arbeit bekommen. Optional können die einzelnen Kapitel der Arbeit kurz vorgestellt und beschrieben werden. Die Problemstellung und der Forschungsstand können in eigenen Unterkapiteln beschrieben werden.
Wichtig: Entweder es gibt mind. 2 Unterkapitel oder kein Unterkapitel! Gibt es nur einen Unterpunkt sollte dieser unbedingt in das Überkapitel mit einbezogen und auf den Unterpunkt verzichtet werden!

\section{Motivation/Problemstellung}
Lorem ipsum dolor sit amet, consetetur sadipscing elitr, sed diam nonumy eirmod tempor invidunt ut labore et dolore magna aliquyam erat, sed diam voluptua. At vero eos et accusam et justo duo dolores et ea rebum. Stet clita kasd gubergren, no sea takimata sanctus est Lorem ipsum dolor sit amet. Lorem ipsum dolor sit amet, consetetur sadipscing elitr, sed diam nonumy eirmod tempor invidunt ut labore et dolore magna aliquyam erat, sed diam voluptua. At vero eos et accusam et justo duo dolores et ea rebum. Stet clita kasd gubergren, no sea takimata sanctus est Lorem ipsum dolor sit amet. 

\section{Zielsetzung}
Lorem ipsum dolor sit amet, consetetur sadipscing elitr, sed diam nonumy eirmod tempor invidunt ut labore et dolore magna aliquyam erat, sed diam voluptua. At vero eos et accusam et justo duo dolores et ea rebum. Stet clita kasd gubergren, no sea takimata sanctus est Lorem ipsum dolor sit amet. Lorem ipsum dolor sit amet, consetetur sadipscing elitr, sed diam nonumy eirmod tempor invidunt ut labore et dolore magna aliquyam erat, sed diam voluptua. At vero eos et accusam et justo duo dolores et ea rebum. Stet clita kasd gubergren, no sea takimata sanctus est Lorem ipsum dolor sit amet.

