\chapter{Abkürzungsverzeichnis}
Das Abkürzungsverzeichnis wird in einer eigenen Datei, die ebenfalls in dem Ordner für Kapitel hinterlegt ist, erstellt. Die Abkürzungen werden hier in diesem Kapitel vorher definiert und können dann einfach im Text aufgerufen werden.

Hier ist \ac{B1} und es gibt auch \ac{B2}. Ruft man ein Beispiel erneut auf, so wird nur noch die Kurzform gezeigt, wie bei \ac{B2} oder \ac{B3}. Außerdem werden in der PDF-Datei nur die Abkürzungen angezeigt, die auch im Text referenziert werden. Im Code ist noch ein \emph{Beispiel 4} hinterlegt. Dieses wird im Verzeichnis jedoch nicht angezeigt, da es im Text nicht referenziert wurde.

Der Text ist für eine Abschlussarbeit selbstverständlich zu löschen. Dieser dient nur zur Erklärung mit dem Umgang des Abkuürzungsverzeichnis.

\begin{acronym}[slmtA]
\acro{B1}{Beispiel 1}
\acro{B2}{Beispiel 2}
\acro{B3}{Beispiel 3}
\acro{B4}{Beispiel 4}
\end{acronym}